% Options for packages loaded elsewhere
\PassOptionsToPackage{unicode}{hyperref}
\PassOptionsToPackage{hyphens}{url}
%
\documentclass[
]{article}
\usepackage{amsmath,amssymb}
\usepackage{iftex}
\ifPDFTeX
  \usepackage[T1]{fontenc}
  \usepackage[utf8]{inputenc}
  \usepackage{textcomp} % provide euro and other symbols
\else % if luatex or xetex
  \usepackage{unicode-math} % this also loads fontspec
  \defaultfontfeatures{Scale=MatchLowercase}
  \defaultfontfeatures[\rmfamily]{Ligatures=TeX,Scale=1}
\fi
\usepackage{lmodern}
\ifPDFTeX\else
  % xetex/luatex font selection
\fi
% Use upquote if available, for straight quotes in verbatim environments
\IfFileExists{upquote.sty}{\usepackage{upquote}}{}
\IfFileExists{microtype.sty}{% use microtype if available
  \usepackage[]{microtype}
  \UseMicrotypeSet[protrusion]{basicmath} % disable protrusion for tt fonts
}{}
\makeatletter
\@ifundefined{KOMAClassName}{% if non-KOMA class
  \IfFileExists{parskip.sty}{%
    \usepackage{parskip}
  }{% else
    \setlength{\parindent}{0pt}
    \setlength{\parskip}{6pt plus 2pt minus 1pt}}
}{% if KOMA class
  \KOMAoptions{parskip=half}}
\makeatother
\usepackage{xcolor}
\usepackage[margin=1in]{geometry}
\usepackage{color}
\usepackage{fancyvrb}
\newcommand{\VerbBar}{|}
\newcommand{\VERB}{\Verb[commandchars=\\\{\}]}
\DefineVerbatimEnvironment{Highlighting}{Verbatim}{commandchars=\\\{\}}
% Add ',fontsize=\small' for more characters per line
\usepackage{framed}
\definecolor{shadecolor}{RGB}{248,248,248}
\newenvironment{Shaded}{\begin{snugshade}}{\end{snugshade}}
\newcommand{\AlertTok}[1]{\textcolor[rgb]{0.94,0.16,0.16}{#1}}
\newcommand{\AnnotationTok}[1]{\textcolor[rgb]{0.56,0.35,0.01}{\textbf{\textit{#1}}}}
\newcommand{\AttributeTok}[1]{\textcolor[rgb]{0.13,0.29,0.53}{#1}}
\newcommand{\BaseNTok}[1]{\textcolor[rgb]{0.00,0.00,0.81}{#1}}
\newcommand{\BuiltInTok}[1]{#1}
\newcommand{\CharTok}[1]{\textcolor[rgb]{0.31,0.60,0.02}{#1}}
\newcommand{\CommentTok}[1]{\textcolor[rgb]{0.56,0.35,0.01}{\textit{#1}}}
\newcommand{\CommentVarTok}[1]{\textcolor[rgb]{0.56,0.35,0.01}{\textbf{\textit{#1}}}}
\newcommand{\ConstantTok}[1]{\textcolor[rgb]{0.56,0.35,0.01}{#1}}
\newcommand{\ControlFlowTok}[1]{\textcolor[rgb]{0.13,0.29,0.53}{\textbf{#1}}}
\newcommand{\DataTypeTok}[1]{\textcolor[rgb]{0.13,0.29,0.53}{#1}}
\newcommand{\DecValTok}[1]{\textcolor[rgb]{0.00,0.00,0.81}{#1}}
\newcommand{\DocumentationTok}[1]{\textcolor[rgb]{0.56,0.35,0.01}{\textbf{\textit{#1}}}}
\newcommand{\ErrorTok}[1]{\textcolor[rgb]{0.64,0.00,0.00}{\textbf{#1}}}
\newcommand{\ExtensionTok}[1]{#1}
\newcommand{\FloatTok}[1]{\textcolor[rgb]{0.00,0.00,0.81}{#1}}
\newcommand{\FunctionTok}[1]{\textcolor[rgb]{0.13,0.29,0.53}{\textbf{#1}}}
\newcommand{\ImportTok}[1]{#1}
\newcommand{\InformationTok}[1]{\textcolor[rgb]{0.56,0.35,0.01}{\textbf{\textit{#1}}}}
\newcommand{\KeywordTok}[1]{\textcolor[rgb]{0.13,0.29,0.53}{\textbf{#1}}}
\newcommand{\NormalTok}[1]{#1}
\newcommand{\OperatorTok}[1]{\textcolor[rgb]{0.81,0.36,0.00}{\textbf{#1}}}
\newcommand{\OtherTok}[1]{\textcolor[rgb]{0.56,0.35,0.01}{#1}}
\newcommand{\PreprocessorTok}[1]{\textcolor[rgb]{0.56,0.35,0.01}{\textit{#1}}}
\newcommand{\RegionMarkerTok}[1]{#1}
\newcommand{\SpecialCharTok}[1]{\textcolor[rgb]{0.81,0.36,0.00}{\textbf{#1}}}
\newcommand{\SpecialStringTok}[1]{\textcolor[rgb]{0.31,0.60,0.02}{#1}}
\newcommand{\StringTok}[1]{\textcolor[rgb]{0.31,0.60,0.02}{#1}}
\newcommand{\VariableTok}[1]{\textcolor[rgb]{0.00,0.00,0.00}{#1}}
\newcommand{\VerbatimStringTok}[1]{\textcolor[rgb]{0.31,0.60,0.02}{#1}}
\newcommand{\WarningTok}[1]{\textcolor[rgb]{0.56,0.35,0.01}{\textbf{\textit{#1}}}}
\usepackage{longtable,booktabs,array}
\usepackage{calc} % for calculating minipage widths
% Correct order of tables after \paragraph or \subparagraph
\usepackage{etoolbox}
\makeatletter
\patchcmd\longtable{\par}{\if@noskipsec\mbox{}\fi\par}{}{}
\makeatother
% Allow footnotes in longtable head/foot
\IfFileExists{footnotehyper.sty}{\usepackage{footnotehyper}}{\usepackage{footnote}}
\makesavenoteenv{longtable}
\usepackage{graphicx}
\makeatletter
\def\maxwidth{\ifdim\Gin@nat@width>\linewidth\linewidth\else\Gin@nat@width\fi}
\def\maxheight{\ifdim\Gin@nat@height>\textheight\textheight\else\Gin@nat@height\fi}
\makeatother
% Scale images if necessary, so that they will not overflow the page
% margins by default, and it is still possible to overwrite the defaults
% using explicit options in \includegraphics[width, height, ...]{}
\setkeys{Gin}{width=\maxwidth,height=\maxheight,keepaspectratio}
% Set default figure placement to htbp
\makeatletter
\def\fps@figure{htbp}
\makeatother
\setlength{\emergencystretch}{3em} % prevent overfull lines
\providecommand{\tightlist}{%
  \setlength{\itemsep}{0pt}\setlength{\parskip}{0pt}}
\setcounter{secnumdepth}{-\maxdimen} % remove section numbering
\ifLuaTeX
  \usepackage{selnolig}  % disable illegal ligatures
\fi
\usepackage{bookmark}
\IfFileExists{xurl.sty}{\usepackage{xurl}}{} % add URL line breaks if available
\urlstyle{same}
\hypersetup{
  pdftitle={Econ 448 Final Paper -- Mexico \& Brazil Marriage Project},
  pdfauthor={Alex Lin},
  hidelinks,
  pdfcreator={LaTeX via pandoc}}

\title{Econ 448 Final Paper -- Mexico \& Brazil Marriage Project}
\author{Alex Lin}
\date{12/12/2025}

\begin{document}
\maketitle

\subsection{Introduction}\label{introduction}

In many models of the family, marriage changes women's incentives to
work by changing specialization inside the household and by changing
each partner's bargaining position. If household production and
childcare can be produced at home, a couple may specialize, with one
partner focusing more on market work while the other focuses more on
non-market production. At the same time, bargaining models predict that
outcomes inside marriage depend on each spouse's ``threat point'' or
outside option, which is shaped by labor market opportunities and
institutions such as divorce law (McElroy and Horney; Chiappori;
Lundberg and Pollak). This paper asks: How is marital status associated
with women's employment and earnings in Mexico and Brazil, and do these
associations differ across countries and over time (around 1990/1991
vs.~2000)?

I use harmonized census microdata from IPUMS International for women
ages 18--65 in Brazil and Mexico. I focus on two core labor outcomes: an
indicator for being employed and log earnings among employed women. My
main finding is that, in both countries, married women are substantially
less likely to be employed than never-married women, and married
employed women have lower earnings than never-married employed women,
even after controlling for age, education, and fertility. The
cross-country interaction results suggest that the ``marriage employment
penalty'' is smaller in Mexico than in Brazil, while the ``divorce
employment penalty'' is broadly similar across countries in this pooled
specification.

\begin{Shaded}
\begin{Highlighting}[]
\FunctionTok{library}\NormalTok{(dplyr)}
\end{Highlighting}
\end{Shaded}

\begin{verbatim}
## 
## 載入套件:'dplyr'
\end{verbatim}

\begin{verbatim}
## 下列物件被遮斷自 'package:stats':
## 
##     filter, lag
\end{verbatim}

\begin{verbatim}
## 下列物件被遮斷自 'package:base':
## 
##     intersect, setdiff, setequal, union
\end{verbatim}

\begin{Shaded}
\begin{Highlighting}[]
\FunctionTok{library}\NormalTok{(ggplot2)}
\FunctionTok{library}\NormalTok{(readr)}
\FunctionTok{library}\NormalTok{(knitr)}
\FunctionTok{library}\NormalTok{(stargazer)}
\end{Highlighting}
\end{Shaded}

\begin{verbatim}
## 
## Please cite as:
\end{verbatim}

\begin{verbatim}
##  Hlavac, Marek (2022). stargazer: Well-Formatted Regression and Summary Statistics Tables.
\end{verbatim}

\begin{verbatim}
##  R package version 5.2.3. https://CRAN.R-project.org/package=stargazer
\end{verbatim}

\begin{Shaded}
\begin{Highlighting}[]
\FunctionTok{library}\NormalTok{(scales)}
\end{Highlighting}
\end{Shaded}

\begin{verbatim}
## 
## 載入套件:'scales'
\end{verbatim}

\begin{verbatim}
## 下列物件被遮斷自 'package:readr':
## 
##     col_factor
\end{verbatim}

\begin{Shaded}
\begin{Highlighting}[]
\DocumentationTok{\#\# 1. Load Data}
\NormalTok{ipums\_file }\OtherTok{\textless{}{-}} \StringTok{"C:/Users/alexl/OneDrive/Desktop/桌面/UW/CourseWork/Senior Year/Autumn 2025/ECON 448/Final Project/ipumsi\_00004.csv.gz"}
\NormalTok{df\_raw }\OtherTok{\textless{}{-}} \FunctionTok{read.csv}\NormalTok{(}\FunctionTok{gzfile}\NormalTok{(ipums\_file))}
\end{Highlighting}
\end{Shaded}

\subsection{Data}\label{data}

The data come from IPUMS International, which provides harmonized
microdata from national censuses in a consistent format across countries
and years. I use Brazil (1991 and 2000) and Mexico (1990 and 2000),
restricting the sample to women ages 18--65. I construct binary
indicators for being ever married, currently married, and divorced, and
I define an employment indicator using EMPSTAT. For earnings, I clean
extremely large income values using the IPUMS top-code and missing value
conventions and analyze log(earnings + 1) to reduce the influence of
skewness. All analysis is conducted at the person level, pooling the
four country-year samples.

\begin{Shaded}
\begin{Highlighting}[]
\DocumentationTok{\#\# 2. Sample Restrictions + Variable Construction}
\NormalTok{df }\OtherTok{\textless{}{-}}\NormalTok{ df\_raw }\SpecialCharTok{\%\textgreater{}\%}
  \FunctionTok{filter}\NormalTok{(}
\NormalTok{    COUNTRY }\SpecialCharTok{\%in\%} \FunctionTok{c}\NormalTok{(}\DecValTok{76}\NormalTok{, }\DecValTok{484}\NormalTok{),}
\NormalTok{    AGE }\SpecialCharTok{\textgreater{}=} \DecValTok{18}\NormalTok{, AGE }\SpecialCharTok{\textless{}=} \DecValTok{65}\NormalTok{,}
\NormalTok{    SEX }\SpecialCharTok{==} \DecValTok{2}
\NormalTok{  ) }\SpecialCharTok{\%\textgreater{}\%}
  \FunctionTok{mutate}\NormalTok{(}
    \AttributeTok{country =} \FunctionTok{ifelse}\NormalTok{(COUNTRY }\SpecialCharTok{==} \DecValTok{484}\NormalTok{, }\StringTok{"Mexico"}\NormalTok{, }\StringTok{"Brazil"}\NormalTok{),}
    \AttributeTok{year =}\NormalTok{ YEAR,}
    \AttributeTok{ever\_married =} \FunctionTok{as.integer}\NormalTok{(MARST }\SpecialCharTok{\textgreater{}=} \DecValTok{2}\NormalTok{),}
    \AttributeTok{married =} \FunctionTok{as.integer}\NormalTok{(MARST }\SpecialCharTok{==} \DecValTok{2}\NormalTok{),}
    \AttributeTok{divorced =} \FunctionTok{as.integer}\NormalTok{(MARST }\SpecialCharTok{==} \DecValTok{4}\NormalTok{),}
    \AttributeTok{employed =} \FunctionTok{as.integer}\NormalTok{(EMPSTAT }\SpecialCharTok{==} \DecValTok{1}\NormalTok{),}
    \AttributeTok{INCEARN\_clean =} \FunctionTok{ifelse}\NormalTok{(INCEARN }\SpecialCharTok{\textgreater{}=} \DecValTok{99999998}\NormalTok{, }\ConstantTok{NA}\NormalTok{, INCEARN),}
    \AttributeTok{ln\_earnings =} \FunctionTok{log}\NormalTok{(INCEARN\_clean }\SpecialCharTok{+} \DecValTok{1}\NormalTok{)}
\NormalTok{  )}
\end{Highlighting}
\end{Shaded}

\subsection{Theory and Predictions}\label{theory-and-predictions}

This project is motivated by two related frameworks from the economics
of the family. First, models of household specialization and household
production predict that marriage can reduce women's labor supply if
couples allocate time efficiently across market work and non-market
production, especially when women bear a larger share of childcare and
home responsibilities. Second, bargaining models predict that
intra-household allocations depend on each spouse's outside options.
When women have stronger outside options, they may obtain a larger share
of household resources and may also change labor supply decisions
because labor market attachment itself can strengthen bargaining power
(McElroy and Horney; Lundberg and Pollak; Chiappori).

These frameworks imply testable predictions. If specialization
dominates, then married women should be less likely to work than
never-married women, holding age and education constant. If bargaining
and outside options dominate, divorced women may work more than married
women because divorce can increase the need for independent income and
because women may invest in labor market attachment to improve outside
options. However, observed differences by marital status are not
automatically causal because marriage and divorce are choices correlated
with unobserved traits, including preferences, health, and local labor
market conditions. This motivates treating the results in this paper as
associations rather than causal effects.

\begin{Shaded}
\begin{Highlighting}[]
\DocumentationTok{\#\# 3. Summary Statistics Table}
\NormalTok{tbl\_sum }\OtherTok{\textless{}{-}}\NormalTok{ df }\SpecialCharTok{\%\textgreater{}\%}
  \FunctionTok{group\_by}\NormalTok{(country, year) }\SpecialCharTok{\%\textgreater{}\%}
  \FunctionTok{summarise}\NormalTok{(}
    \AttributeTok{N =} \FunctionTok{n}\NormalTok{(),}
    \AttributeTok{mean\_AGE =} \FunctionTok{mean}\NormalTok{(AGE, }\AttributeTok{na.rm =} \ConstantTok{TRUE}\NormalTok{),}
    \AttributeTok{sd\_AGE =} \FunctionTok{sd}\NormalTok{(AGE, }\AttributeTok{na.rm =} \ConstantTok{TRUE}\NormalTok{),}
    \AttributeTok{min\_AGE =} \FunctionTok{min}\NormalTok{(AGE, }\AttributeTok{na.rm =} \ConstantTok{TRUE}\NormalTok{),}
    \AttributeTok{max\_AGE =} \FunctionTok{max}\NormalTok{(AGE, }\AttributeTok{na.rm =} \ConstantTok{TRUE}\NormalTok{),}
    \AttributeTok{mean\_YRSCHOOL =} \FunctionTok{mean}\NormalTok{(YRSCHOOL, }\AttributeTok{na.rm =} \ConstantTok{TRUE}\NormalTok{),}
    \AttributeTok{sd\_YRSCHOOL =} \FunctionTok{sd}\NormalTok{(YRSCHOOL, }\AttributeTok{na.rm =} \ConstantTok{TRUE}\NormalTok{),}
    \AttributeTok{mean\_CHBORN =} \FunctionTok{mean}\NormalTok{(CHBORN, }\AttributeTok{na.rm =} \ConstantTok{TRUE}\NormalTok{),}
    \AttributeTok{sd\_CHBORN =} \FunctionTok{sd}\NormalTok{(CHBORN, }\AttributeTok{na.rm =} \ConstantTok{TRUE}\NormalTok{),}
    \AttributeTok{mean\_employed =} \FunctionTok{mean}\NormalTok{(employed, }\AttributeTok{na.rm =} \ConstantTok{TRUE}\NormalTok{),}
    \AttributeTok{mean\_ln\_earn =} \FunctionTok{mean}\NormalTok{(ln\_earnings, }\AttributeTok{na.rm =} \ConstantTok{TRUE}\NormalTok{),}
    \AttributeTok{.groups =} \StringTok{"drop"}
\NormalTok{  )}

\FunctionTok{kable}\NormalTok{(}
\NormalTok{  tbl\_sum,}
  \AttributeTok{digits =} \DecValTok{3}\NormalTok{,}
  \AttributeTok{caption =} \StringTok{"Summary statistics for women ages 18–65 by country and year"}
\NormalTok{)}
\end{Highlighting}
\end{Shaded}

\begin{longtable}[]{@{}
  >{\raggedright\arraybackslash}p{(\columnwidth - 24\tabcolsep) * \real{0.0625}}
  >{\raggedleft\arraybackslash}p{(\columnwidth - 24\tabcolsep) * \real{0.0391}}
  >{\raggedleft\arraybackslash}p{(\columnwidth - 24\tabcolsep) * \real{0.0625}}
  >{\raggedleft\arraybackslash}p{(\columnwidth - 24\tabcolsep) * \real{0.0703}}
  >{\raggedleft\arraybackslash}p{(\columnwidth - 24\tabcolsep) * \real{0.0547}}
  >{\raggedleft\arraybackslash}p{(\columnwidth - 24\tabcolsep) * \real{0.0625}}
  >{\raggedleft\arraybackslash}p{(\columnwidth - 24\tabcolsep) * \real{0.0625}}
  >{\raggedleft\arraybackslash}p{(\columnwidth - 24\tabcolsep) * \real{0.1094}}
  >{\raggedleft\arraybackslash}p{(\columnwidth - 24\tabcolsep) * \real{0.0938}}
  >{\raggedleft\arraybackslash}p{(\columnwidth - 24\tabcolsep) * \real{0.0938}}
  >{\raggedleft\arraybackslash}p{(\columnwidth - 24\tabcolsep) * \real{0.0781}}
  >{\raggedleft\arraybackslash}p{(\columnwidth - 24\tabcolsep) * \real{0.1094}}
  >{\raggedleft\arraybackslash}p{(\columnwidth - 24\tabcolsep) * \real{0.1016}}@{}}
\caption{Summary statistics for women ages 18--65 by country and
year}\tabularnewline
\toprule\noalign{}
\begin{minipage}[b]{\linewidth}\raggedright
country
\end{minipage} & \begin{minipage}[b]{\linewidth}\raggedleft
year
\end{minipage} & \begin{minipage}[b]{\linewidth}\raggedleft
N
\end{minipage} & \begin{minipage}[b]{\linewidth}\raggedleft
mean\_AGE
\end{minipage} & \begin{minipage}[b]{\linewidth}\raggedleft
sd\_AGE
\end{minipage} & \begin{minipage}[b]{\linewidth}\raggedleft
min\_AGE
\end{minipage} & \begin{minipage}[b]{\linewidth}\raggedleft
max\_AGE
\end{minipage} & \begin{minipage}[b]{\linewidth}\raggedleft
mean\_YRSCHOOL
\end{minipage} & \begin{minipage}[b]{\linewidth}\raggedleft
sd\_YRSCHOOL
\end{minipage} & \begin{minipage}[b]{\linewidth}\raggedleft
mean\_CHBORN
\end{minipage} & \begin{minipage}[b]{\linewidth}\raggedleft
sd\_CHBORN
\end{minipage} & \begin{minipage}[b]{\linewidth}\raggedleft
mean\_employed
\end{minipage} & \begin{minipage}[b]{\linewidth}\raggedleft
mean\_ln\_earn
\end{minipage} \\
\midrule\noalign{}
\endfirsthead
\toprule\noalign{}
\begin{minipage}[b]{\linewidth}\raggedright
country
\end{minipage} & \begin{minipage}[b]{\linewidth}\raggedleft
year
\end{minipage} & \begin{minipage}[b]{\linewidth}\raggedleft
N
\end{minipage} & \begin{minipage}[b]{\linewidth}\raggedleft
mean\_AGE
\end{minipage} & \begin{minipage}[b]{\linewidth}\raggedleft
sd\_AGE
\end{minipage} & \begin{minipage}[b]{\linewidth}\raggedleft
min\_AGE
\end{minipage} & \begin{minipage}[b]{\linewidth}\raggedleft
max\_AGE
\end{minipage} & \begin{minipage}[b]{\linewidth}\raggedleft
mean\_YRSCHOOL
\end{minipage} & \begin{minipage}[b]{\linewidth}\raggedleft
sd\_YRSCHOOL
\end{minipage} & \begin{minipage}[b]{\linewidth}\raggedleft
mean\_CHBORN
\end{minipage} & \begin{minipage}[b]{\linewidth}\raggedleft
sd\_CHBORN
\end{minipage} & \begin{minipage}[b]{\linewidth}\raggedleft
mean\_employed
\end{minipage} & \begin{minipage}[b]{\linewidth}\raggedleft
mean\_ln\_earn
\end{minipage} \\
\midrule\noalign{}
\endhead
\bottomrule\noalign{}
\endlastfoot
Brazil & 1991 & 4659362 & 35.666 & 12.766 & 18 & 65 & 5.650 & 7.677 &
7.768 & 21.240 & 0.375 & 10.246 \\
Brazil & 2000 & 5962845 & 36.334 & 12.847 & 18 & 65 & 7.364 & 10.273 &
2.478 & 2.663 & 0.446 & 5.081 \\
Mexico & 1990 & 2120963 & 34.268 & 12.679 & 18 & 65 & 7.000 & 10.693 &
8.433 & 21.761 & 0.234 & 2.853 \\
Mexico & 2000 & 2780233 & 35.320 & 12.731 & 18 & 65 & 10.203 & 17.675 &
4.176 & 10.580 & 0.333 & 2.101 \\
\end{longtable}

\subsection{Descriptive Results}\label{descriptive-results}

\subsection{Summary Statistics}\label{summary-statistics}

Table 1 reports summary statistics by country and year. Several broad
patterns stand out. First, employment levels rise over time in both
countries: Brazil increases from about 0.375 employed in 1991 to 0.446
in 2000, and Mexico increases from about 0.234 in 1990 to 0.333 in 2000.
Second, average years of schooling are higher in later years for both
countries, consistent with educational expansion. Finally, the earnings
variable is expressed in local currency units and therefore differs
mechanically across countries; cross-country comparisons of log earnings
should be interpreted cautiously unless income is converted into a
comparable unit (for example, PPP-adjusted values). For this reason, the
earnings analysis is best interpreted as reflecting within-country
differences by marital status, with country and year controls absorbing
level differences.

\begin{Shaded}
\begin{Highlighting}[]
\DocumentationTok{\#\# 4. Figure: Proportion Ever Married by Age}
\NormalTok{df\_ever }\OtherTok{\textless{}{-}}\NormalTok{ df }\SpecialCharTok{\%\textgreater{}\%}
  \FunctionTok{group\_by}\NormalTok{(country, year, AGE) }\SpecialCharTok{\%\textgreater{}\%}
  \FunctionTok{summarise}\NormalTok{(}
    \AttributeTok{prop\_ever =} \FunctionTok{mean}\NormalTok{(ever\_married, }\AttributeTok{na.rm =} \ConstantTok{TRUE}\NormalTok{),}
    \AttributeTok{.groups =} \StringTok{"drop"}
\NormalTok{  )}

\FunctionTok{ggplot}\NormalTok{(df\_ever, }\FunctionTok{aes}\NormalTok{(}\AttributeTok{x =}\NormalTok{ AGE, }\AttributeTok{y =}\NormalTok{ prop\_ever, }\AttributeTok{color =} \FunctionTok{factor}\NormalTok{(year), }\AttributeTok{group =}\NormalTok{ year)) }\SpecialCharTok{+}
  \FunctionTok{geom\_line}\NormalTok{() }\SpecialCharTok{+}
  \FunctionTok{facet\_wrap}\NormalTok{(}\SpecialCharTok{\textasciitilde{}}\NormalTok{ country) }\SpecialCharTok{+}
  \FunctionTok{scale\_y\_continuous}\NormalTok{(}\AttributeTok{labels =} \FunctionTok{percent\_format}\NormalTok{(}\AttributeTok{accuracy =} \DecValTok{1}\NormalTok{)) }\SpecialCharTok{+}
  \FunctionTok{labs}\NormalTok{(}
    \AttributeTok{x =} \StringTok{"Age"}\NormalTok{,}
    \AttributeTok{y =} \StringTok{"Proportion ever married"}\NormalTok{,}
    \AttributeTok{color =} \StringTok{"Year"}\NormalTok{,}
    \AttributeTok{title =} \StringTok{"Proportion of Women Ever Married by Age, Country, and Year"}
\NormalTok{  ) }\SpecialCharTok{+}
  \FunctionTok{theme\_bw}\NormalTok{()}
\end{Highlighting}
\end{Shaded}

\includegraphics{Econ-448-Final-Paper_files/figure-latex/unnamed-chunk-5-1.pdf}

\subsection{Ever-married profiles}\label{ever-married-profiles}

Figure 1 plots the proportion of women who have ever been married by
age, separately by country and year. In both countries, ever-marriage
rises steeply from the late teens through the 20s and then flattens at
older ages. The overall shape is similar across years within each
country, suggesting that marriage timing is broadly stable in this
period, though the exact levels differ slightly across year and country.
This descriptive pattern helps contextualize later regressions by
showing that marital status is strongly age-related, reinforcing the
importance of controlling flexibly for age in the regression models.

\subsection{Empirical Strategy}\label{empirical-strategy}

I estimate two main models. For employment, I use a linear probability
model:

\[
\text{Employed}_i
= \alpha
+ \beta_1 \text{Married}_i
+ \beta_2 \text{Divorced}_i
+ f(\text{Age}_i)
+ \gamma \text{Schooling}_i
+ \delta \text{Children}_i
+ \text{Country FE}
+ \text{Year FE}
+ \varepsilon_i .
\]

where \(f(\text{Age}_i)\) is a quadratic in age. I then extend this to
allow the relationship between marital status and employment to differ
in Mexico relative to Brazil by interacting marital status indicators
with the Mexico dummy.

For earnings, I restrict the sample to employed women with non-missing
earnings and estimate:

\[
\ln(\text{Earnings}_i + 1)
= \alpha
+ \beta_1 \text{Married}_i
+ \beta_2 \text{Divorced}_i
+ f(\text{Age}_i)
+ \gamma \text{Schooling}_i
+ \delta \text{Children}_i
+ \text{Country FE}
+ \text{Year FE}
+ \varepsilon_i .
\]

These models are designed to test whether outcomes differ systematically
by marital status after controlling for key observable correlates.
Because the data are cross-sectional and marital status is endogenous,
the estimates should be interpreted as associations rather than causal
effects.

\subsection{Regression Results}\label{regression-results}

\begin{Shaded}
\begin{Highlighting}[]
\DocumentationTok{\#\# 5. Employment Regressions}
\NormalTok{m\_emp1 }\OtherTok{\textless{}{-}} \FunctionTok{lm}\NormalTok{(}
\NormalTok{  employed }\SpecialCharTok{\textasciitilde{}}\NormalTok{ married }\SpecialCharTok{+}\NormalTok{ divorced }\SpecialCharTok{+}
\NormalTok{    AGE }\SpecialCharTok{+} \FunctionTok{I}\NormalTok{(AGE}\SpecialCharTok{\^{}}\DecValTok{2}\NormalTok{) }\SpecialCharTok{+}\NormalTok{ YRSCHOOL }\SpecialCharTok{+}\NormalTok{ CHBORN }\SpecialCharTok{+}
\NormalTok{    country }\SpecialCharTok{+} \FunctionTok{factor}\NormalTok{(year),}
  \AttributeTok{data =}\NormalTok{ df}
\NormalTok{)}

\NormalTok{m\_emp2 }\OtherTok{\textless{}{-}} \FunctionTok{lm}\NormalTok{(}
\NormalTok{  employed }\SpecialCharTok{\textasciitilde{}}\NormalTok{ married}\SpecialCharTok{*}\NormalTok{country }\SpecialCharTok{+}\NormalTok{ divorced}\SpecialCharTok{*}\NormalTok{country }\SpecialCharTok{+}
\NormalTok{    AGE }\SpecialCharTok{+} \FunctionTok{I}\NormalTok{(AGE}\SpecialCharTok{\^{}}\DecValTok{2}\NormalTok{) }\SpecialCharTok{+}\NormalTok{ YRSCHOOL }\SpecialCharTok{+}\NormalTok{ CHBORN }\SpecialCharTok{+}
    \FunctionTok{factor}\NormalTok{(year),}
  \AttributeTok{data =}\NormalTok{ df}
\NormalTok{)}

\FunctionTok{stargazer}\NormalTok{(}
\NormalTok{  m\_emp1, m\_emp2,}
  \AttributeTok{title =} \StringTok{"Association Between Marital Status and Employment"}\NormalTok{,}
  \AttributeTok{dep.var.labels =} \StringTok{"Employed"}\NormalTok{,}
  \AttributeTok{column.labels =} \FunctionTok{c}\NormalTok{(}\StringTok{"Pooled"}\NormalTok{, }\StringTok{"With interactions"}\NormalTok{),}
  \AttributeTok{covariate.labels =} \FunctionTok{c}\NormalTok{(}
    \StringTok{"Married"}\NormalTok{,}
    \StringTok{"Divorced"}\NormalTok{,}
    \StringTok{"Married × Mexico"}\NormalTok{,}
    \StringTok{"Divorced × Mexico"}\NormalTok{,}
    \StringTok{"Age"}\NormalTok{,}
    \StringTok{"Age squared"}\NormalTok{,}
    \StringTok{"Years of schooling"}\NormalTok{,}
    \StringTok{"Children ever born"}\NormalTok{,}
    \StringTok{"Mexico (country dummy)"}
\NormalTok{  ),}
  \AttributeTok{keep =} \FunctionTok{c}\NormalTok{(}
    \StringTok{"married$"}\NormalTok{,}
    \StringTok{"divorced$"}\NormalTok{,}
    \StringTok{"married:countryMexico"}\NormalTok{,}
    \StringTok{"countryMexico:divorced"}\NormalTok{,}
    \StringTok{"AGE$"}\NormalTok{,}
    \StringTok{"I}\SpecialCharTok{\textbackslash{}\textbackslash{}}\StringTok{(AGE}\SpecialCharTok{\textbackslash{}\textbackslash{}}\StringTok{\^{}2}\SpecialCharTok{\textbackslash{}\textbackslash{}}\StringTok{)"}\NormalTok{,}
    \StringTok{"YRSCHOOL$"}\NormalTok{,}
    \StringTok{"CHBORN$"}\NormalTok{,}
    \StringTok{"countryMexico$"}
\NormalTok{  ),}
  \AttributeTok{omit.stat =} \FunctionTok{c}\NormalTok{(}\StringTok{"f"}\NormalTok{, }\StringTok{"ser"}\NormalTok{),}
  \AttributeTok{digits =} \DecValTok{3}\NormalTok{,}
  \AttributeTok{type =} \StringTok{"text"}
\NormalTok{)}
\end{Highlighting}
\end{Shaded}

\begin{verbatim}
## 
## Association Between Marital Status and Employment
## ===================================================
##                            Dependent variable:     
##                        ----------------------------
##                                  Employed          
##                          Pooled   With interactions
##                           (1)            (2)       
## ---------------------------------------------------
## Married                -0.261***      -0.231***    
##                         (0.0003)      (0.0003)     
##                                                    
## Divorced               -0.152***      -0.165***    
##                         (0.001)        (0.001)     
##                                                    
## Married × Mexico        0.043***      0.043***     
##                         (0.0001)      (0.0001)     
##                                                    
## Divorced × Mexico      -0.001***      -0.001***    
##                        (0.00000)      (0.00000)    
##                                                    
## Age                     0.004***      0.004***     
##                        (0.00001)      (0.00001)    
##                                                    
## Age squared            -0.0004***    -0.0005***    
##                        (0.00001)      (0.00001)    
##                                                    
## Years of schooling     -0.108***      -0.042***    
##                         (0.0003)       (0.001)     
##                                                    
## Children ever born                    -0.101***    
##                                        (0.001)     
##                                                    
## Mexico (country dummy)                0.037***     
##                                        (0.001)     
##                                                    
## ---------------------------------------------------
## Observations           15,523,403    15,523,403    
## R2                       0.109          0.111      
## Adjusted R2              0.109          0.111      
## ===================================================
## Note:                   *p<0.1; **p<0.05; ***p<0.01
\end{verbatim}

\subsection{Employment results}\label{employment-results}

Table 2 shows a large and statistically precise association between
marital status and employment. In the pooled model, being married is
associated with a 26.1 percentage point lower probability of being
employed relative to never-married women, and being divorced is
associated with a 15.2 percentage point lower probability. In the
interaction specification, the baseline (Brazil) married coefficient is
about -0.231, while the married interaction for Mexico is +0.043,
implying that the employment gap between married and never-married women
is still negative in Mexico but smaller than in Brazil by about 4.3
percentage points. The divorced interaction for Mexico is essentially
zero in magnitude, suggesting the divorce-employment association is
similar across countries in this pooled model.

These findings are consistent with household specialization models in
which marriage is associated with reduced market work for women. They
are also consistent with bargaining approaches if marriage reduces
women's market attachment or if women's outside options differ
systematically by marital status (McElroy and Horney; Lundberg and
Pollak; Chiappori). That said, selection into marriage is likely
important: women with lower labor market attachment may be more likely
to marry, and unobserved preferences and local labor market conditions
could bias these estimates.

\begin{Shaded}
\begin{Highlighting}[]
\DocumentationTok{\#\# 6. Earnings Regression (Employed Women Only)}
\NormalTok{df\_working }\OtherTok{\textless{}{-}}\NormalTok{ df }\SpecialCharTok{\%\textgreater{}\%}
  \FunctionTok{filter}\NormalTok{(employed }\SpecialCharTok{==} \DecValTok{1}\NormalTok{, }\SpecialCharTok{!}\FunctionTok{is.na}\NormalTok{(ln\_earnings))}

\NormalTok{m\_earn }\OtherTok{\textless{}{-}} \FunctionTok{lm}\NormalTok{(}
\NormalTok{  ln\_earnings }\SpecialCharTok{\textasciitilde{}}\NormalTok{ married }\SpecialCharTok{+}\NormalTok{ divorced }\SpecialCharTok{+}
\NormalTok{    AGE }\SpecialCharTok{+} \FunctionTok{I}\NormalTok{(AGE}\SpecialCharTok{\^{}}\DecValTok{2}\NormalTok{) }\SpecialCharTok{+}\NormalTok{ YRSCHOOL }\SpecialCharTok{+}\NormalTok{ CHBORN }\SpecialCharTok{+}
\NormalTok{    country }\SpecialCharTok{+} \FunctionTok{factor}\NormalTok{(year),}
  \AttributeTok{data =}\NormalTok{ df\_working}
\NormalTok{)}

\FunctionTok{stargazer}\NormalTok{(}
\NormalTok{  m\_earn,}
  \AttributeTok{title =} \StringTok{"Association Between Marital Status and Log Earnings (Employed Women Only)"}\NormalTok{,}
  \AttributeTok{dep.var.labels =} \StringTok{"log(earnings + 1)"}\NormalTok{,}
  \AttributeTok{covariate.labels =} \FunctionTok{c}\NormalTok{(}
    \StringTok{"Married"}\NormalTok{,}
    \StringTok{"Divorced"}\NormalTok{,}
    \StringTok{"Age"}\NormalTok{,}
    \StringTok{"Age squared"}\NormalTok{,}
    \StringTok{"Years of schooling"}\NormalTok{,}
    \StringTok{"Children ever born"}\NormalTok{,}
    \StringTok{"Mexico (country dummy)"}\NormalTok{,}
    \StringTok{"Year dummies"}
\NormalTok{  ),}
  \AttributeTok{digits =} \DecValTok{3}\NormalTok{,}
  \AttributeTok{omit.stat =} \FunctionTok{c}\NormalTok{(}\StringTok{"f"}\NormalTok{, }\StringTok{"ser"}\NormalTok{),}
  \AttributeTok{type =} \StringTok{"text"}
\NormalTok{)}
\end{Highlighting}
\end{Shaded}

\begin{verbatim}
## 
## Association Between Marital Status and Log Earnings (Employed Women Only)
## ==================================================
##                            Dependent variable:    
##                        ---------------------------
##                             log(earnings + 1)     
## --------------------------------------------------
## Married                         -0.592***         
##                                  (0.002)          
##                                                   
## Divorced                        -0.052***         
##                                  (0.005)          
##                                                   
## Age                             0.172***          
##                                  (0.001)          
##                                                   
## Age squared                     -0.002***         
##                                 (0.00001)         
##                                                   
## Years of schooling              0.026***          
##                                 (0.0001)          
##                                                   
## Children ever born              -0.005***         
##                                 (0.0001)          
##                                                   
## Mexico (country dummy)          1.416***          
##                                  (0.003)          
##                                                   
## Year dummies                    -0.810***         
##                                  (0.004)          
##                                                   
## factor(year)2000                -6.004***         
##                                  (0.004)          
##                                                   
## Constant                        8.289***          
##                                  (0.010)          
##                                                   
## --------------------------------------------------
## Observations                    5,723,772         
## R2                                0.609           
## Adjusted R2                       0.609           
## ==================================================
## Note:                  *p<0.1; **p<0.05; ***p<0.01
\end{verbatim}

\subsection{Earnings results}\label{earnings-results}

Table 3 reports results for log earnings among employed women. Married
employed women have substantially lower earnings than never-married
employed women: the married coefficient is about -0.592, which
corresponds to roughly 45\% lower earnings in a log interpretation ( 𝑒 −
0.592 − 1 ≈ − 0.447 e −0.592 −1≈−0.447). Divorced employed women also
have lower earnings than never-married employed women, but the magnitude
is much smaller (about -0.052, or roughly 5\%). Education is positively
related to earnings, and the age profile is concave, consistent with a
typical earnings lifecycle pattern.

In the context of family economics, the married earnings penalty among
employed women could reflect several mechanisms: differential selection
into employment (for example, married women may be more likely to work
intermittently or in lower-paying jobs compatible with household
production), employer discrimination, or a ``motherhood penalty''
channel correlated with marriage and fertility. Bargaining models do not
necessarily predict higher earnings for married women; instead, they
predict that allocations respond to outside options and institutions,
and labor market behavior can be part of how outside options are formed
(Lundberg and Pollak; Chiappori). Because earnings are measured in local
currency and the pooled model includes a country dummy, cross-country
comparisons of earnings levels are not meaningful without conversion;
the key takeaway here is the within-sample marital status gradient.

\subsection{Discussion and
Limitations}\label{discussion-and-limitations}

This paper's results strongly suggest that marital status is correlated
with women's labor market outcomes in both Mexico and Brazil. The
descriptive patterns show that marriage is nearly universal by midlife
and closely linked to age, while the regression estimates show that the
married category is associated with lower employment and lower earnings
among workers.

However, there are important limitations. First, the estimates are not
causal: marriage and divorce are endogenously chosen and correlated with
unobserved characteristics. Second, earnings comparisons across
countries are complicated by different currency units and potentially
different top-coding rules; a cleaner cross-country earnings comparison
would convert incomes into a common metric such as PPP-adjusted dollars
or would analyze within-country standardized earnings measures. Third,
the census data provide limited information on job characteristics (for
example, hours, informality, or sector) that may mediate the
relationship between marriage and earnings. Finally, the extremely large
sample sizes mean very small differences can be statistically
significant, so substantive significance should be emphasized alongside
p-values.

Even with these limitations, the results line up with well-established
ideas in the family economics literature: marriage is associated with
changes in time allocation and labor market attachment, and these
relationships can differ across settings depending on social norms,
labor markets, and institutions governing family formation and
dissolution (McElroy and Horney; Chiappori; Lundberg and Pollak).
Research on divorce law changes also highlights that institutions can
shape labor supply responses, though identifying those effects requires
policy variation and careful causal designs (Hoehn-Velasco and Penglase;
Stevenson and Wolfers).

\subsection{Conclusion}\label{conclusion}

Using IPUMS International census microdata for Brazil and Mexico around
1990/1991 and 2000, I find that marital status is strongly associated
with women's employment and earnings. Married women are much less likely
to be employed than never-married women, and among employed women,
marriage is associated with substantially lower earnings. A pooled
interaction model suggests the marriage-employment gap is somewhat
smaller in Mexico than in Brazil, while the divorced-employment
relationship is similar across the two countries in this specification.
Overall, the patterns are consistent with family economics theories
emphasizing specialization and bargaining, while also highlighting the
importance of selection and institutional context. Future work could
improve cross-country earnings comparability and move toward causal
identification using policy variation in family law.

\newpage

Works Cited

Chiappori, Pierre-André. ``Collective Labor Supply and Welfare.''
Journal of Political Economy, vol.~100, no. 3, 1992, pp.~437--467.
University of Chicago Press, \url{https://doi.org/10.1086/261825}

Hoehn-Velasco, Lauren, and Jacob Penglase. ``Does Unilateral Divorce
Impact Women's Labor Supply? Evidence from Mexico.'' Journal of Economic
Behavior \& Organization, vol.~187, 2021, pp.~315--347. Elsevier,
\url{https://doi.org/10.1016/j.jebo.2021.04.028}

IPUMS International. IPUMS International: Version 7.7 {[}dataset{]}.
Minneapolis, MN: IPUMS, 2025. \url{https://doi.org/10.18128/D020.V7.7}

Lundberg, Shelly, and Robert A. Pollak. ``Separate Spheres Bargaining
and the Marriage Market.'' American Economic Review, vol.~83, no. 4,
1993, pp.~988--1010.

McElroy, Marjorie B., and Mary Jean Horney. ``Nash-Bargained Household
Decisions: Toward a Generalization of the Theory of Demand.''
International Economic Review, vol.~22, no. 2, 1981, pp.~333--349.

Stevenson, Betsey, and Justin Wolfers. ``Bargaining in the Shadow of the
Law: Divorce Laws and Family Distress.'' Quarterly Journal of Economics,
vol.~121, no. 1, 2006, pp.~267--288.

\end{document}
